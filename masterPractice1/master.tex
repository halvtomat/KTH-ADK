\documentclass[11pt]{article}
\usepackage{listings}

\title{DD2350 Övningsmästarprov 1, Kortaste texten}
\date{2020-09-21}
\author{Daniel Gustafsson}

\begin{document}
\pagenumbering{gobble}  
\maketitle
\newpage

\section{Pseudokod}
\begin{lstlisting}
    int antalRader(int n, int len, int w[])
    int rader = 1
    int pos = 0

    for(i = 0 to n)
        pos+=w[i]
        if(pos > len)
            rader++
            pos = w[i]
        else pos++
    return rader
\end{lstlisting}
\section{Tidskomplexitetsanalys}
Enhetskostnaden för en iteration är O(1) och består av 3,
eller i vissa iterationer 4, triviala operationer.

Algoritmen kommer iterera igenom 0 - n utan undantag och 
har därför en tidskomplexitet på O(n).
\section{Korrekthetsanalys}
För att bevisa algoritmens korrekthet kan man använda
loop-invarianter. Exempel på loop-invarianter som stämmer
för denna algoritm är att "rader" variabeln stämmer
efter varje iteration. Det betyder alltså att "rader" har
värdet av det optimala antalet rader för de orden 
algoritmen har itererarat igenom hittills efter varje 
iteration.
Om ovanstående loop-invariant stämmer skulle det bevisa
att algoritmen alltid returnerar det optimala antalet rader.
\end{document}